\documentclass[10pt]{amsart}
\usepackage[top=1.5in, bottom=1.25in, left=1.25in, right=1.25in]{geometry}
\usepackage[backend=biber]{biblatex}

\usepackage{mathtools}
\usepackage{hyperref}
\usepackage{amsmath}
\usepackage{amsthm}
\usepackage[]{algorithm2e}
\RestyleAlgo{boxed}
\usepackage{algpseudocode}
\usepackage{listings}
\usepackage{appendix}
\usepackage{courier}

\newtheorem{theorem}{Theorem}[section]
\newtheorem{exmp}{Example}[section]
\newtheorem{defn}{Definition}[section]

\bibliography{Stats}

\begin{document}
\title{Statistics in SpliceMachine}
\author{Scott Fines}

\begin{abstract}
				Statistics are a core part of improving the performance of any database of reasonable size. They are used during query optimization to choose the best execution plans, during the execution of plans to manage physical resources, and by resource managers to choose optimal distributions across multiple nodes. This document serves as a design construction for how Statistics should be implemented in SpliceMachine.
\end{abstract}

\maketitle

Understanding the role of statistics in a relational database is typically tightly connected to undertstanding the role of the query optimizer. In most databases\footnote{See Appendix-\ref{sec:OtherDBs}}, Statistics information is a tool which is used only during the query planning and optimization stage, and is not used at any other stage of the execution process. This view that statistics is only helpful to the query optimizer has significant consequences on how the statistical systems in those databases have been implemented. There are variations, but the central theme of statistics collections in database products to this point is to acquire a small but statistically significant sample of data, and computing a simple set of statistics from this sample. And, because sampling eliminates all boundaries between data partitions (either within a node or between multiple nodes), this approach has the circular effect of forcing database systems to view data only in a global manner\footnote{Or, as in the case with Oracle, to draw a strict distinction between global and local statistics}.

We propose an adjustment to this paradigm. Clearly, the query optimizer will need to maintain a global understanding of data, in order to choose the correct plans. However, in addition to that, SpliceMachine will maintain statistics within boundaries (in particular, within the boundaries of regions). This will allow SpliceMachine to make optimization choices which are not available to the query planner, and therefore to improve throughput and reduce latency by well chosen execution strategies. Further, this approach will lead to a better use of resources;from dynamically sizing in-memory buffers to choosing single-threaded versus concurrent implementations of the same algorithm, we can use statistical information to ensure that SpliceMachine runs more stably and more predictably even under the stress of many different queries.

\section{Overview}
There are two main components of the statistics strategy in SpliceMachine:

\begin{description}
\item[Collection] approaches what statistics are collected and how the collection will occur.
\item[Distribution and Storage] resolves issues around storing statistical data, and how it can be accessed, both for human administrators and internal systems.
\end{description}

\subsection{Collection}
Statistics will be collected through the use of a periodically executed maintenance tasks, which can be triggered on a single region in one of two modes. Manual mode (which will be implemented first), will collect statistics when the administrator issues the appropriate request. There will be some variations on those procedures to ensure that statistics can be collected with varying degrees of thoroughness (and with varying performance characteristics).

Automatic mode, by contrast, will attempt to refresh statistics whenever the system can reasonably detect that a refresh would be helpful. In particular, whenever a region detects that it has received a significant number of successful writes, it will assume that it needs new statistics, and will submit a task for execution. However, this maintenance task will wait to be executed until the region detects a substantial decrease in it's overall write load (Should such a situation never occur, it will eventually execute anyway).

As automatic mode may choose to use resources at an inopurtune time, it will come with an associated disable call, which an administrator can use to disable automatic collection for high-load databases.

\subsection{Distribution and Storage}
Once collected for a given region, Statistics must be made available for the query optimizers on \emph{all} RegionServers to use. 

The most obvious way to do this is to store the data into an HBase table. In particular, there will be two tables: 

\begin{description}
				\item[\texttt{sys\_table\_statistics}] maintains statistics for the table as a whole, including latency and other physical statistics.
				\item[\texttt{sys\_column\_statistics}] maintains statistics for each individual column
\end{description}

Each region will update specific rows in this table, making the storage format region-specific. This allows individual regions to update themselves without requiring the entire system to update.

These two tables will hold data in binary formats which are efficient to store and can be combined correctly. This has the unfortunate consequence that these tables will not be human readable in any direct sense. To convert between the binary storage points and the human-readable table constructs, we will use \emph{virtual table interfaces}(VTIs) to create two views of the tables:

\begin{description}
				\item[\texttt{sys\_table\_stats}] will be a human-readable view of the table as a whole
				\item[\texttt{sys\_column\_stats}] will be a human-readable view of for each individual column
\end{description}

VTI views will allow users to treat these views exactly as if they were tables, with one exception: these tables (and their views) will be considered read-only; modification of statistics is allowed only by the statistics engine itself.

There is a significant downside to this strategy that we must deal with. We note that table access is a relatively expensive operation to conduct remotely. This is made more difficult when the data may only be present on a single node \footnote{Hotspotting regions is a common HBase performance problem, after all}.  However, we are helped by the realization that statistics updates will occur relatively rarely in our system\footnote{Rarely in this case can be anywhere between once a month for a reference data set to once every hour for high-volume OLTP tables}. Because of this, we can safely cache statistics data locally on each RegionServer on the cluster.

In order to avoid memory (and CPU) penalties in this cache, caching will have three levels. The first level is for regions which are known to reside on a given RegionServer. Those regions will cache their data at a regional level, which will allow them to use region-specific statistics to perform runtime optimizations. 

The second level is the remote region level, where statistics for regions which are managed remotely are held. This level of caching is relatively small, since there are relatively few circumstances under which we may wish for statistics for remote regions.

The third level is the global table level--statistics are merged during table reading, and only a single view of the table as a whole is maintained in the cache. When the cache is refreshed, the entire table's statistics will be read from tables and updated. This will avoid a potentially expensive merging process during query optimization and other situations where a global view is desired.

\section{Statistics}
The strategy that we've outlined so far has expressed itself in the singular "Statistics", but what, exactly, are these statistics that we plan on collecting, and why are they considered useful for us?

Roughly speaking, we would like to collect anything(and everything) that will give us more information about the data that is stored. We are tolerant to some degree of error in our estimates, as long as we can get accurate enough that we make the correct decisions during optimization. 

We can safely group our statistics into two distinct categories: \emph{logical} and \emph{runtime}.  

\subsection{Logical Statistics}
Logical Statistics are what we generally think of when we think of statistics; they involve information that we can use to resolve a (slighty fuzzy) view of the data without requiring reference to the data itself.

We separate our logical statistics into two distinct subcategories. On the global side, we maintain \emph{table}-level statistics, which provide us with information about the table as a whole, ignoring specific columns. For more precise information, we also maintain \emph{column}-level statistics, which provide us with information about the columns themselves.

\subsection{Table-Level Statistics}
Table-level statistics are reasonably simple to maintain and understand, as they deal with rows in the absence of additional column information. 

We maintain the following set of statistics:

\begin{enumerate}
				\item Row Count
				\item Total Size of table in bytes
				\item Mean width of a Row in bytes (including RowKey)
				\item Mean width of a RowKey in bytes
				\item Total Query Count
				\item Region Count
				\item Mean Region Size in bytes
				\item Distribution of Rows by Region as Histogram
				\item Distribution of Query Count by Region as Histogram
				\item Server Count
				\item Mean "Server Size" in bytes
				\item Distribution of Rows by Server as Histogram
				\item Distribution of Query Count by Server as Histogram
\end{enumerate}

Region and Server Counts measure the number of Regions(and the number of Servers, respectively) which are actively involed in managing this table's data. For example, if the table has ten regions scattered over 3 servers, then the Region Count is ten and the Server Count is three.

The additional fields attempt to keep track of the distribution of data on a per-region and per-server basis. This is represented easiest as a histogram distribution of data by server and region. The distribution of rows allows the load balancer and/or administrators to quickly determine which regions contain an excessively large number of rows, while the distribution of query counts will allow the load balancer and administrators to determine "hot" regions and servers--regions and servers which are involved with an excessively large volume of traffic.

As a note, it is not always strictly necessary to keep all this information stored on disk--for example, the region count can be acquired directly, without reference to any tables.

\subsection{Column Statistics}
Column Statistics manage information about individual columns, and are often the more interesting (and more difficult) algorithms to implement. While table-level metrics are easy to acquire\footnote{In fact, one could simply maintain the correct values for all table-level metrics and have a reasonable degree of accuracy}, column statistics are significantly more complex.

\subsubsection{Cardinality}
The first, and simplest, metric that we wish to record is the \emph{cardinality}. Mathematically, \emph{cardinality} is just the number of entries in a set. As we are dealing with multisets(sets which can contain more than one entry with the same value), cardinality is more appropriately stated as the number of \emph{distinct} elements in the data set. 

There are a number of algorithms which can estimate the cardinality of a data set with varying degrees of accuracy. However, the most effective estimation strategy is \emph{HyperLogLog}\cite{Flajolet07hyperloglog:the}. For details on the algorithm itself, see Appendix-\ref{sec:HyperLogLog}.

HyperLogLog essentially strikes a balance between memory consumption and accuracy. A more accurate estimate requires more counters,and thus more total memory. However, once a given accuracy is specified, the space requirements are \emph{constant} with respect to the number of rows processed. Additionally, updates to the data structure are constant-time. 

In practice, each counter requires only a byte of space, so the total memory space is the cost of a single object, a single array, and $2^b$ bytes. If $b=14$, this memory cost is $\approx 16$ kilobytes per column(Heule et. al's adjustments can reduce this cost for low cardinalities\cite{HyperLogLogGoogle}). Thus, over the maximum 1024 columns, the total memory footprint due to cardinality checking is $\approx 2^{b+10}$ bytes(for $b=14$, this is $\approx 16$ megabytes).

\subsubsection{Frequent Elements}
The next set of statistics to collect for SpliceMachine is referred to as the \emph{frequent elements}\footnote{Also referred to as the \emph{Heavy Hitters}, or the \emph{Iceberg Values}}. Frequent elements are simple the elements which occur most frequently in the data set; collecting them and their exact frequencies allows us to be more accurate for queries in which these frequent elements are involved.

There are fewer algorithms available for solving the Frequent elements problem, but there are still a reasonably large number to parse through. However, the \emph{SpaceSaver} algorithm stands out as a particularly effective choice.

The SpaceSaver algorithm keeps an approximate list of heavy hitters using constant space (controlled by the desired accuracy). It does this by keeping not just a fixed-size list of elements, but also a fixed-size list of error estimates, which it can use to eliminate errneously counted heavy hitters\footnote{We elide the details of the algorithm in this section. For those details, see Appendix-\ref{sec:SpaceSaver}}. As a result of the structural designs and a bit of clever data structure organization, SpaceSaver is able to allow constant-time updates with a bounded constant-space memory cost.

\subsubsection{Histograms}
The core of any reasonable statistics structure is the \emph{histogram}. Roughly speaking, a histogram is a summary structure which shows the distribution of ordered data across it's possible domain. Typically, it is used in data visualization tools, but when the need to make visual sense is removed, histograms are allowed to become more exotic in favor of better estimating capacities.

We will avoid discussing histograms in too much depth during this section--for more detail, see Appendix-\ref{sec:Histograms}. However, there are a few crucial points which must be discussed in order to have a full understanding of the problem that Histograms present.

Histograms come in three major forms\footnote{There are actually four or five additional forms, but they amount to variations on the main three}: \emph{Equi-width},\emph{Equi-depth}, and \emph{V-Optimal}.

Equi-width Histograms are usually what people imagine when they think of Histograms: the domain of values is split into a certain number of \emph{buckets}, each of which has a fixed, equal share of the domain of values. Some buckets will have more values than others, but all elements will belong to one and only one bucket. 

Equi-width histograms are trivial to implement, and multiple equi-width histograms can be easily merged together, making for simple distribution. However, they also display poor estimation quality\footnote{In the interest of brevity, details are not provided in this section. See Appendix-\ref{sec:EquiDepth} for evidence in support of these claims.}.

A better choice in terms of error estimation (particular for query optimizer applications) is the \emph{Equi-Depth} histogram. Equi-depth histograms also subdivide the domain of values into a fixed number of buckets; however, the boundaries of these buckets are chosen so that all buckets have (approximately) the same number of entries. In practice, these are quite effective at providing reasonable information for query optimizers to use\footnote{In fact, all databases surveyed in Appendix-\ref{sec:OtherDBs} use an Equi-depth histogram}. 

However, Equi-depth histograms suffer from a significant downside: they cannot be merged without violating the equi-depth histogram constraint (and thus greatly increasing the estimation error). As a result, it is impossible for SpliceMachine to provide an Equi-depth histogram that has both region-level information and global information.

This leaves V-Optimal histograms. a V-Optimal histogram is similar to an Equi-depth histogram, in that it uses a fixed number of buckets, each of which are of variable widths. Unlike an Equi-depth histogram, however, V-Optimal histograms attempt to set bucket boundaries such that the variance between items inside each individual bucket is minimized. This results in the most accurate histogram \footnote{for the purposes of query optimization, at any rate}, but is significantly more difficult to construct. As if the difficulty weren't enough, V-Optimal histograms share the nonlinear characteristics of Equi-depth histograms, making them a poor choice for SpliceMachine.

Thus, at first glance, we are confronted with a tough choice: do we sacrifice our desire for a region-level statistics view so that we may have the accuracy benefits of an Equi-depth of V-Optimal histogram, or do we sacrifice accuracy in our statistics structures so that we may easily merge together histograms?

Fortunately, there is a third choice we can make. \emph{Wavelet transforms} are a method of approximating a function using a linear sum of other functions(analogously to Fourier Transforms, although not precisely the same). They are commonly used in signal processing applications where data is considered to be a stream of incoming values whose contribution to the whole is not predictable.

It is only a slight change to transform our way of thinking from a static \emph{set} of data points which must be described using a histogram, into one where we are a \emph{stream} of changes to a \emph{signal}, where the signal in this case is the \emph{cumulative density function} of the data itself--in other words, the signal is the histogram, and we attempt to approximate that signal using Wavelet transforms.

The elegance of Wavelet transforms are that they are linear (and therefore easy to merge\cite{GilbertWaveletHistograms}), and accurate--much more so than random samplig\cite{MatiasWaveletHistogram}. 

The most straightforward algorithm for computing wavelent-based histograms is the \emph{GroupedCount Sketch}\cite{CormodeGroupedCountSketch}, which uses roughly $O(\log{n})$ space and requires roughtly $O(\log{n})$ steps to update.

\subsection{Summary}
For our mental convenience, we summarize here which column statistics are to be collected:

\begin{enumerate}
				\item Cardinality
				\item \emph{Null Fraction}
				\item Frequent Elements
				\item Frequency estimate for Frequent Elements
				\item Wavelet Histogram (if data type is ordered)
				\item Minimum Value (if data type is ordered)
				\item Maximum values (if data type is ordered)
\end{enumerate}

where the \emph{Null Fraction} is defined as the number of null values, divided by the total number of rows.

\subsection{Physical Statistics}
In addition to logical statistics, we must collect some basic information about the physical world in which we are operating. Because we are durable, a large component of our cost is the cost of reading data off disk; because we are a clustered environment, the second largest power is the cost to write and read data over the network. Thus, we will need to collect disk and network I/O latency. 

In other systems (notably Oracle\cite{Oracle}), information about hardware is collected at startup; this is a good approach for collecting localized metrics that cannot change (such as number of disks, etc.), but is not applicable to all metrics. Thus, we have two categories: \emph{fixed} and \emph{variable} measurements.

Fixed measurements consist of:

\begin{enumerate}
				\item Number of CPUs
				\item Max Heap Size
				\item Max Block Cache Size
				\item Number of IPC threads
\end{enumerate}

while variable measurements are:
\begin{enumerate}
				\item Local Read Latency
				\item Remote Read Latency
				\item Write network latency (to TEMP)
\end{enumerate}

Fixed measurements are collected once during startup and recorded, while variable measurements need to be periodically refreshed. 

The initial way of collecting variable measurements is to use the logical statistics gathering phase. During the statistics collection phase, take a uniform sample of rows from the table, and write those rows to TEMP. If the table being measured is an index table, use the same sample to perform a simple index lookup on the main table, which will provide index lookup and write latency measures. Local read measures can be acquired merely by recording the read performance of the statistics gathering process.

Remote read latency is more difficult to measure. Initially, it will be recorded during some transaction lookups, although ultimately a more sophisticated method may be used.

When automatic collection is enabled, a good way of obtaining these measurements is similar to that of PostgreSQL\cite{PGCollector} measures real time performance, and is referred to as \emph{query sampling}. When query sampling is enabled, a random sample of queries is taken\footnote{The sampling logic may be selective--for example, it may only only randomly sample from those queries which are expected to take a very short period of time, so that the added cost is not excessive.}. When a query is chosen, it will record the latency measurements that occurred during the execution of that query. This would allow more accurate variable measurements, but would have an adverse impact on performance for selected queries, so it would also require a shut-off valve to disable it when performance is critical.

\section{Collection}
Now that we have a clear image of what, precisely, we are collecting within our statistics engine, it behooves us to talk about the collection mechanism itself.

There will ultimately be two modes of collections: \emph{manual} and \emph{automatic}. Manual mode will be implemented first, as it is simpler to implement and test, while automatic mode will be initiated afterwards.

\subsection{Manual Collection Mode}
Manual collection mode consists of a set of stored procedures, which will be used to initiate statistics gathering in a variety of different ways.

First, we will allow granularity of gathering, so that an administrator is not required to initiate collection for the entire database every time. In particular, we will have the following stored procedures:

\begin{description}
				\item[\texttt{COLLECT\_ALL\_STATISTICS()}] Will collect statistics for every table and index in the database. This will likely be excessively expensive.
				\item[\texttt{COLLECT\_SCHEMA\_STATISTICS('SCHEMA')}] Will collect statistics for every table and index in the specified \texttt{SCHEMA}. 
				\item[\texttt{COLLECT\_TABLE\_STATISTICS('SCHEMA','TABLE')}] Will collect statistics for the specified table or index.
				\item[\texttt{COLLECT\_COLUMN\_STATISTICS('SCHEMA','TABLE','COLUMNS')}] Will collect statistics for one or more specified columns in the particular table.
								\item[\texttt{COLLECT\_REGION\_STATISTICS('SCHEMA','TABLE','start key','stop key')}] For primary keyed tables and indices, this will collect statistics for a specified region of a table. This is useful for forcing statistics collection after a manual split, for example.
\end{description}
This will allow administrators some control over the resource usage of the collection tasks. Note, however, that collection always occurs within a maintenance task frame, so the number of available tasks for execution are limited.

\subsection{Automatic Collection Mode}
Ultimately, an automatic collection mode will be an effective tool. When automatic collection is enabled, an individual region will keep watch on how many mutations are received. When a threshold of mutations has been reached, it will schedule a new gathering task for itself, to be executed \emph{after} the rate of mutations received has decreased to a smaller volume. This way, individual regions can proactively adjust their statistics information without interfering excessively with ongoing writes. 

These tasks will be queued in such a way as to guarantee that

\begin{enumerate}
				\item No Region may collect statistics more than a fixed number of times in a given period of time.
				\item No more than one statistics task for a region can be queued at a time--others tasks are merged in to the process.
\end{enumerate}

\subsection{Transactions}
The question arises as to what isolation level the Statistics engine should use during collection tasks. Each level exposes us to a different type of potential error. 

If we choose read uncommitted, then we will see all active and committed records, and thus we will collect statistics in the most optimistic sense (that active transactions will eventually become committed). However, if a large transaction were to be rolled back, then we would incorrectly include those rows in the statistics view, leading us to believe that the table is larger than it truly is.

If we choose read committed or repeatable reads, then we will not see transactions that are currently active, only rows which belong to committed transactions. Thus, if we collected statistics during a large insert or import, we would not have those records, leading us to incorrectly assume that the table is smaller than it truly is.

Our choice is made a bit simpler when we consider how transactional structures work in SpliceMachine. In order for a given row to be filtered transactionally, it must first be read off disk--thus, the most expensive component in the transactional process must occur no matter what. This means that every operation will functionally read every row that was durably written, even if that row belongs to a transaction which has been rolled back. Therefore, we would like to include that read cost in our statistical model, which implies that we want to include rows at any possible stage. However, we would \emph{not} want to include rolled back rows, as they could potentially destroy the predictive abilities of our intermediate estimates. 

Thus, after consideration, it seems clear that we should use a read uncommitted isolation level when collecting statistics.

\section{Storage}
We would like our statistics to be distributed in a straightforward and durable manner. The simplest way to do this is using an HBase table for storage. Then, when statistics are needed, they can simply be read.

In order to support the region-level isolation that our statistics engine requires, we will need to also separate writes out by individual regions. 

\subsection{Table Statistics}
Table statistics are stored in the \texttt{sys\_table\_statistics} table, whose schema is as in Table-\ref{table:tableStatistics}; it has an effective primary key of (\texttt{schema\_name},\texttt{table\_name},\texttt{region\_id}).

\begin{table}
				\begin{tabular}{|l|c|p{6cm}|}
								\hline
								\bf{Name}									& \bf{Type}	&	\bf{Description} \\ \hline	
								\texttt{schema\_name}			&	varchar	&	The Schema of the table \\ \hline
								\texttt{table\_name}			&	varchar	&	The name of the table \\ \hline
								\texttt{conglom\_id}			&	bigint	&	The conglomerate id for the table \\ \hline
								\texttt{region\_id}				&	varchar	&	the region identifier \\ \hline
								\texttt{num\_rows}				&	bigint	& RowCount \\ \hline
								\texttt{region\_size}			&	bigint	&	Total size of region (in bytes) \\ \hline
								\texttt{mean\_row\_width}	&	integer	&	mean row width (in bytes) \\ \hline
								\texttt{mean\_key\_width}	&	integer	&	mean key width (in bytes) \\ \hline
								\texttt{query\_count}			&	bigint	&	Total number of queries issued \\ \hline
				\end{tabular}
				\caption{\texttt{sys\_table\_statistics} table schema}
				\label{table:tableStatistics}
\end{table}

Note that some table-level statistics (such as server information) are not stored. This is because the location of regions may change frequently, and it is relatively efficient for the system to collect that data on demand rather than storing data which is probably out of data anyway.

Because \texttt{sys\_table\_statistics} is not directly managed by derby, we expose a read-only VTI view called \texttt{sys\_table\_stats} which is exposed through derby, to allow administrators direct query access to the table.

\begin{table}
				\begin{tabular}{|l|c|p{6cm}|}
								\hline
								\bf{Name}														& \bf{Type}	&	\bf{Description} \\ \hline	
								\texttt{schema\_name}								&	varchar		&	The Schema of the table \\ \hline
								\texttt{table\_name}								&	varchar		&	The name of the table \\ \hline
								\texttt{num\_rows}									&	bigint		&	The number of rows in the table \\ \hline
								\texttt{mean\_row\_width}						&	integer		&	The mean width of a row (in bytes) 	\\ \hline
								\texttt{mean\_key\_width}						&	integer		&	The mean width of a key (in bytes) \\ \hline
								\texttt{mean\_region\_size}					&	bigint		&	The mean size of the region (in bytes) \\ \hline
								\texttt{min\_region\_size}					&	bigint		&	The minimum region size (in bytes) \\ \hline
								\texttt{max\_region\_size}					&	bigint		&	The maximum region size (in bytes) \\ \hline
								\texttt{num\_regions}								&	integer		&	Region count	\\	\hline
								\texttt{mean\_region\_query\_count}	&	bigint		&	The mean number of queries per region \\ \hline
								\texttt{max\_region\_query\_count}	&	bigint		&	The max number of queries per region \\ \hline
								\texttt{min\_region\_query\_count}	&	bigint		&	The min number of queries per region \\ \hline
								\texttt{num\_servers}								&	integer		&	Server count \\ \hline
								\texttt{mean\_server\_size}					&	bigint		&	The mean size per server (in bytes) \\ \hline
								\texttt{min\_server\_size}					&	bigint		&	The minimum size per server (in bytes) \\ \hline
								\texttt{max\_server\_size}					&	bigint		&	The maximum size per server (in bytes) \\ \hline
								\texttt{mean\_server\_query\_count}	&	bigint		&	The mean number of queries per server  \\ \hline
								\texttt{max\_server\_query\_count}	&	bigint		&	The max number of queries per server \\ \hline
								\texttt{min\_server\_query\_count}	&	bigint		&	The min number of queries per server \\ \hline
				\end{tabular}
				\caption{\texttt{sys\_table\_stats} table schema}
				\label{table:tableStats}
\end{table}
\subsection{Column Statistics}
Column Statistics are stored in the \texttt{sys\_column\_statistics} table, which is an internal-only table containing binary representations of column statistics algorithms. The table has a schema as described in Table-\ref{table:columnStatistics}, and has an effective primary key of 
\linebreak(\texttt{conglomerate\_id},\texttt{column\_num},\texttt{region\_id}).

\begin{table}
				\begin{tabular}{|l|c|p{6cm}|}
								\hline
								\bf{Name}									& \bf{Type}	&	\bf{Description} \\ \hline	
								\texttt{conglomerate\_id}	&	bigint		&	The conglomerate id for the table \\ \hline
								\texttt{region\_id}				&	varchar		&	The region identifier \\ \hline
								\texttt{column\_num}			&	smallint	& The column number \\ \hline
								\texttt{cardinality}			&	binary		&	cardinality estimate \\ \hline
								\texttt{null\_frac}				&	real			&	null fraction \\ \hline
								\texttt{freq\_vals}				&	binary		&	encoded list of most frequent values \\ \hline
								\texttt{freq\_freqs}			&	binary		&	encoded list of frequencies for most common values \\ \hline
								\texttt{wavelet\_coeffs}	&	binary		&	encoded list of wavelet coefficients \\ \hline
								\texttt{min\_val}					&	column type	&	the minimum column value	\\	\hline
								\texttt{max\_val}					&	column type	&	the maximum column value	\\	\hline
				\end{tabular}
				\caption{\texttt{sys\_column\_statistics} table schema}
				\label{table:columnStatistics}
\end{table}

In order to expose the binary structures in the column stats, we expose a VTI view \linebreak\texttt{sys\_column\_stats} with the schema as in Table-\ref{table:columnStats}. This VTI table is primarily intended for user exploration and visibility more than for functional usability within the system.

\begin{table}
				\begin{tabular}{|l|c|p{6cm}|}
								\hline
								\bf{Name}									& \bf{Type}	&	\bf{Description} \\ \hline	
								\texttt{conglomerate\_id}	&	bigint		&	The conglomerate id for the table \\ \hline
								\texttt{column\_num}			&	smallint	&	The column number \\ \hline
								\texttt{cardinality}			&	bigint		&	The cardinality estimate \\ \hline
								\texttt{null\_frac}				&	real			& Null fraction \\ \hline
								\texttt{freq\_vals}				&	varchar		&	comma-separated list of frequent values, and their frequencies \\ \hline	
								\texttt{wavelet}					&	varchar		&	functional representation of the wavelent coefficients \\ \hline
								\texttt{min\_val}					&	column type	&	the minimum column value \\ \hline
								\texttt{max\_val}					&	column type	& the maximum column value \\ \hline
				\end{tabular}
				\caption{\texttt{sys\_column\_stats} table schema}
				\label{table:columnStats}
\end{table}

\subsection{Physical Statistics}
Physical statistics are not durably stored, because there are not many values, and they are server-specific rather than table-specific. Instead, physical statistics will be periodically refreshed into an in-memory view. When a refresh is desired, the engine will randomly select a server from a known list of active servers, and will request a refresh of that data. When data is not available for a specific server, the engine will assume that it has a similar set of metrics as another server in the cluster; thus, it will just use a mean value for everything.

In order to view these metrics on a per-server level, we expose a VTI table \texttt{sys\_physical\_stats}, which has a schema as in Table-\ref{table:physicalStats}.

\begin{table}
				\begin{tabular}{|l|c|p{6cm}|}
								\hline
								\bf{Name}													& \bf{Type}	&	\bf{Description} \\ \hline	
								\texttt{server\_ip}								&	varchar		&	IP address of server \\ \hline
								\texttt{hostname}									&	varchar		&	Hostname of server \\ \hline
								\texttt{num\_cpus}								&	integer		& Number of CPUs \\ \hline
								\texttt{max\_heap\_size}					&	integer		&	Max heap size (in bytes)\\ \hline
								\texttt{max\_block\_cache\_size}	&	integer		&	Max block cache size (in bytes) \\ \hline
								\texttt{num\_ipc\_threads}				&	integer		&	Number of IPC threads \\ \hline
								\texttt{local\_read\_latency}			&	bigint		&	Local read latency (ns) \\ \hline
								\texttt{remote\_read\_latency}		&	bigint		& Remote read latency (ns) \\ \hline
								\texttt{write\_latency}						&	bigint		&	Remote write latency (ns) \\ \hline
				\end{tabular}
				\caption{\texttt{sys\_physical\_stats} table schema}
				\label{table:physicalStats}
\end{table}

\subsection{Internal Caching}
It is not desirable for internal optimization tools to perform a remote table lookup in order to acquire statistics that aren't likely to have changed much. Because statistics are rarely changed\footnote{"rarely", in this case, can mean anywhere between months and minutes in between changes,depending on configuration}, and because stale statistics are not a significant problem for SpliceMachine, it is possible to maintain a \emph{Statistics Cache} on each node, which will have three levels:

\begin{enumerate}
				\item Local Regions Cache
				\item Remote Regions Cache
				\item Table Cache
\end{enumerate}

All caches except for the Local Regions Cache will be periodically refreshed with a background table scan, which will fetch the latest information for that specific entry. 

\subsubsection{Local Regions Cache}
Local regions will hold their statistics in the Local Region cache, and every time statistics are updated, they are also written to the Local Regions Cache. As a result, this cache is never stale. When accessing statistics for those particular regions, the Local Regions cache is always consulted, thus avoiding any table reads.

\subsubsection{Remote Regions Cache}
On the other hand, when a region is not accessible, but region-level statistics are still desired, the \emph{Remote Regions Cache} will be consulted. This will maintain a small cache of remote Region statistics, which can be used to avoid a table access during remote region statistics gathering.

As the usefulness of remote regions is not generally significant, the Remote Regions cache is likely to be very small.

\subsubsection{Table Cache}
When statistics for the entire table are desired in a global, merged sense, a \emph{Table Cache} will be maintained, which will allow the caller to quickly access global statistics for a table without requiring an expensive scan and merge process.

\clearpage
\begin{subappendices}
				\label{sec:Appendix}
				\section*{Appendix: Statistics in Other Database Products}
				\label{sec:OtherDBs}
				Statistics are commonly used in most database products, and have been since the early days of System R. Primarily, these statistics have been oriented towards query planning and optimization, although some databases allow the use of statistics in the field of approximate query answering\footnote{This is significantly more rare, and usually involves additional work beyond just looking at pre-collected statistics information}. 

\subsection{Oracle}
Oracle is a single-node database(with available replication styles), so it's focus is on a "global" view of the data under management. In this sense, there is only one level of granularity on statistics--that of a table.  Additionally, because of its single-node architecture, Oracle emphasizes the minimization of disk I/O and CPU (as opposed to disk I/O and network, as in the case of SpliceMachine); this focus is reflected in the statistics which it collects.

Oracle collects statistics using a periodic update algorithm, where a single explicit operation collects logical statistics for a database,table, or specific columns. Inside this operation, a random sample of rows are taken, and this sample is used to build all logical statistics for the table. The size of this sample determines the accuracy of the resulting statistics, and is configured manually (with reasonable defaults).

This periodic update can be triggered in two separate ways: Manually, or automatically. When automatic collection is enabled, Oracle will attempt to schedule a background collection operation during periods of known low activity. This is sufficient for common use cases where there \emph{is} a regular period of downtime, and only moderate changes to data occur in between updates. However, if the contents of a table change very quickly, or there is no period during which resources can be devoted to statistics collection, automatic collection poses a performance and stability bottleneck. To address this, Oracle allows automatic collection to be disabled. When automatic mode is disabled, it is the responsibility of the system administrator to ensure that statistics are properly collected.

Oracle collects statistics in four distinct groups\cite{Oracle}:
\begin{enumerate}
				\item Table
								\begin{enumerate}
												\item	Row Count
												\item Number of Blocks
												\item Average Row Length (in bytes)
								\end{enumerate}
				\item Column
								\begin{enumerate}
												\item Average Column Length (in bytes)
												\item Cardinality
												\item Number of null values
												\item Equi-depth Histogram
								\end{enumerate}
				\item Index
								\begin{enumerate}
												\item Number of Leaf Blocks
												\item Levels
												\item Clustering Factor
								\end{enumerate}
				\item Physical
								\begin{enumerate}
												\item I/O performance and utilization
												\item CPU performance and utilization
								\end{enumerate}
\end{enumerate}
Statistics on the number of blocks and leaf-blocks are present because of Oracle's internal B-Tree structure.

\subsubsection{Partitioned Tables}
Oracle supports \emph{partitioned tables}, which are similar in effect to SpliceMachine's regions (although not distributed across multiple machines). To support these within their statistical engine, Oracle may optionally collect statistics for each individual partition in addition to collecting for the table as a whole\cite{Oracle}. This model is similar to that of SpliceMachine, except for one key difference. In Oracle, there is no sense of unity between the statistics of the partition and the statistics of the entire table. Each set of statistics is collected and used independently of the other, which requires consumers to use either the partitioned results \emph{or} the global results, but not both simultaneously.

\subsubsection{Oracle RAC}
Oracle RAC is Oracle's distributed version of their Oracle product, and so it has some relevance to SpliceMachine. 

Oracle RAC is a shared-storage environment\cite{OracleRAC}. Because of this, any node in the Oracle RAC environment has the same access to all data, and thus can compute statistics exactly as if that node operated in a single-node environment. As a result, OracleRAC has the same statistics scheme as single-node Oracle instances do.

\subsection{PostgreSQL}
PostgreSQL has a very similar architecture to that of Oracle, and thus it adopts many of the same strategies for statistics collection and usage. In particular, PostgreSQL makes use to statistics only for query optimization, and operates in a single-node environment where B-Trees are the underlying storage structure. This leads PostgreSQL to collect and use the following statistics\cite{PGCollector,PGStats,PGClass}:
\begin{enumerate}
				\item Table
								\begin{enumerate}
												\item Row Count
												\item Page Count
								\end{enumerate}
				\item Column
								\begin{enumerate}
												\item Null Fraction
												\item Average Column Width
												\item Cardinality
												\item Most Frequent Values
												\item Frequencies of Most Frequent Values
												\item Equi-Depth Histogram
								\end{enumerate}
				\item Physical Statistics
								\begin{enumerate}
												\item I/O latency
												\item CPU utilization
								\end{enumerate}
\end{enumerate}
as well as a number of monitoring statistics (lock hold time, CPU usage time and so on).

As with Oracle, Postgres supports both automatic and manual collection modes. When automatic collection is enabled, large operations (such as bulk imports, index creation, and vacuum processes) will automatically trigger a statistics update; additionally, a periodic statistics collection process is engaged regularly to keep statistics as up-to-date as possible. As part of this process, a random sample of queries will collect runtime statistics information, in order to keep physical statistics up to date. 

Also similar to Oracle, PostgreSQL allows automatic collection mode to be disabled for performance and resource management purposes.

\subsubsection{Master-Master replication}
PostgreSQL allows a master-master replication solution, which performs replication between two active master nodes\cite{PGReplication}. This solution does not perform sharding, merely replication, so all data that is present on a single master node is also present on all other nodes in the cluster. While this is technically shared-nothing, each node in a PostgreSQL cluster has access to the same data, so each node can generate statistics as if it were in a single-node environment.

\subsection{SQL Server}
Microsoft's SQL Server product is very similar in features and architecture to Oracle and PostgreSQL, which makes SQL Server's statistics collection very similar as well. In particular, SQLServer follows the lead of Oracle and PostgreSQL in allowing both manual and automatic collection modes, with an ability to disable automatic collection as needed.

The statistics collected by SQLServer are\cite{SQLServerStats}:

\begin{enumerate}
	\item Table
					\begin{enumerate}
									\item Row Count
									\item Page Count
					\end{enumerate}
	\item Column
					\begin{enumerate}
									\item Average Width (in bytes)
									\item histogram containing:
													\begin{enumerate}
																	\item Max Value in Range
																	\item Rows in Range
																	\item Rows equal to Max Value in Range
																	\item Average number of rows per unique value in range
																	\item Cardinality of range
													\end{enumerate}
					\end{enumerate}
\end{enumerate}

\subsection{EMC Greenplum}
EMC Greenplum makes use of a distributed, shared-nothing architecture that is similar in many ways to that chosen by SpliceMachine.

Unfortunately, very little public information is available regarding the statistics collector in particular; one must assemble a rough estimation based on other facts.

Greenplum is based off of PostgreSQL, so it is reasonable to assume that the statistics which are collected are similar to that of PostgreSQL. Additionally, it makes extensive use of the Scatter-gather operation to generate intermediate results\cite{GreenplumAdmin}. It's likely then that Greenplum uses this technique to generate a sample of data\footnote{The author would like to emphasize his total lack of knowledge regarding Greenplum--corrections and addendums would be welcome}, which it uses to compute a single global view of the distribution of data.

One thing that is noteworthy is that all planning in Greenplum occurs on the master, and is performed globally\cite{GreenplumAdmin}. It is therefore unlikely that statistics are used at the segment level for additional optimization. This eliminates the need for segment isolation that SpliceMachine desires.

\subsection{Vertica}
Vertica is a unique shared-nothing architecture in that it emphasizes a columnar storage format instead of the row-based storage of Oracle et al. This has some consequences on its statistics engine.

Vertica collects the following statistics\cite{Vertica}:

\begin{enumerate}
				\item Table
								\begin{enumerate}
												\item	Row Count
								\end{enumerate}
				\item Column
								\begin{enumerate}
												\item	Cardinality
												\item Min and Max values
												\item Equi-Depth Histogram
												\item Disk Space occupied
								\end{enumerate}
\end{enumerate}

Because of its columnar structure, Vertica does not have need of a significant number of table statistics.

Only global statistics are created, and computing these statistics involves all nodes in the cluster. In order to compute statistics, a random sample of entries is computed, which is either $2^{17}$ rows or $1GB$ of space, whichever is smaller. Statistics are then computed from this sample directly\cite{Vertica}.

It is not directly explained why $2^{17}$ rows are chosen, or $1GB$ of space, but it is reasonable to assume that this limit is applied so that a single machine may compute global statistics.

Vertica uses statistics only for query planning and optimization; physical optimizations and load balancing do not involve statistical information. 

As with Oracle and others, Vertica has both manual and automatic collection modes; it is possible to disable automatic collection mode via configuration.


				\clearpage
				\section*{Appendix: Algorithms for Statistics}
				\label{sec:Algorithms}
				\section{Preliminaries}
In order to understand the algorithms stated, it is helpful to first agree on terminology, and some essential definitions.

\subsection{Hashing and Uniform Hash functions}
Hashing plays a critical role in several algorithms which we will discuss, so it's helpful to rehash\footnote{pun intended} some introductory computer science information for reference.

A $p$-bit hash function is simply a function $h$ which takes as input an element from the data set, and emits a $p$-bit scalar value. Thus, a 32-bit hash function emits 32-bit integers, while a 64-bit hash function emits 64-bit scalars.

By extension, a \emph{uniform hash function} is a $p$-bit hash function $h$ which ensures that the bits which are output are uniformly distributed; that is, that the resulting $p$-bit scalar is uniformly distributed across all possible $p$-bit scalars.

\subsection{Linear functions}
A large focus of Statistics research in SpliceMachine orients around how to make multiple independent systems generate statistics which are capable of being merged together--that is, that two individual regions can generate a stastitics estimate which can be combined together in such a way that no\footnote{or very little} additional error is introduced.

Mathematically, this is equivalent to saying that a function is \emph{linear}. More precisely

\begin{defn}
				Suppose $R$ is a data set, and suppose that $\psi(R)$ is a function. $\psi$ is considered \emph{linear} if 
				\begin{displaymath}
								\psi(R_1 \cup R_2) = \psi(R_1) + \psi(R_2)
				\end{displaymath}
				for any two independent data sets $R_1$ and $R_2$.
\end{defn}

In other words, two independent regions will be able to collect statistics independently if and only if the statistics function is linear. 

\section{Cardinality}
\label{sec:HyperLogLog}
\emph{Cardinality} is the measure of the number of unique elements in a a data set. For example, if the data set consists of $\lbrace 1,1,4,-2,3,6,6,7,8,10\rbrace$, then the unique elements are $\lbrace 1,4,-2,3,6,7,8,10\rbrace$, and the cardinality is 8. 

There are a number of different algorithms for estimating the cardinality of a data set, but one of the most effective method is \emph{HyperLogLog}\cite{Flajolet07hyperloglog:the}. 

Assume that there is a $k$-bit uniform hash function $h$\footnote{In practice, $k = 64$ in pretty much all cases}. Then we know that, for any $x$ in the data set, each bit of $h(x)$ is equally likely to be 1 or 0. Thus, the probability that the bit at position $p$ is 1 is $\approx 2^{-p}$. Thus, over the entire data set, $\approx n/2^p$ entries will be hashed such that the bit at position $p$ is set(where $n$ is the number of distinct entries in the set). As a result, we can say that $n \approx 2^p$.

In a perfect world, this would be all that is necessary. Unfortunately, that approximation has an unbounded error\footnote{technically, it has \emph{infinite expectation}, which means that the sequence doesn't converge.} so we won't be able to make use of just a single estimate. 

However, if we hash entries into multiple buckets, and perform the above estimate for each bucket, and then average the resulting buckets, we can correct for the error growth problem and have a very good estimate for our cardinality.

Thus, HyperLogLog accepts a parameter $b$, which is an integer assesment of the amount of error to accept. The higher the value of $b$, the more accurate the estimations become for higher cardinalities. Then we create $2^b$ independent counters. As a new entity comes in, it is hashed into a single $k$-bit scalar. The first $b$ bits of that scalar is the id for the counter to use, and we define $\rho$ to be the index of the first set bit in the remaining $k-b$ bits. The counter value holds the minimum of its current contents and $\rho$. Then, to compute the cardinality estimate, we take a specific average over all the counters.

\begin{algorithm}[ht]
				\KwData{ let $M$ be a dataset,$h$ be a $k$-bit uniform hash function,$b$ be an error control parameter, and $\alpha$ an empirically determined constant depending on $b$}
				\KwResult{$C = $ the estimated cardinality of $M$}
				\Begin{
					\For{$v \in M$}{
							set $x = h(v)$ \;
							set $j = 1+ <x_1x_2...x_b>_2$ the first $b$ bits of $x$ \;
							set $w = <x_{b+1}x_{b+2}...>_2$ the remaining $k-b$ bits of $x$\;
							set $\rho =$ the position of the leftmost 1-bit in $w$ \;
							set $c[j] = max(c[j],\rho)$ \;
						}
						\Return{
										$\alpha2^{2b}(\sum_{j=1}^{2^b}{2^{-c[j]}})^{-1}$
						}
		}
				\caption{HyperLogLog Algorithm}
\end{algorithm}

We can note that the maximum value of any single counter is $k-b$; for a 64-bit hash function, and $b\geq 4$, this means a maximum value of 60 for any single counter. Hence, we can use a single byte to represent each counter.

This algorithm is very accurate for large cardinalities. In fact, it can be proven that the error is probabilistically bounded by $\sigma = 1.04/\sqrt{2^b}$; the error will be less than $\sigma$ 65\% of the time, $2\sigma$ 95\% of the time, and $3\sigma$ 99\% of the time. This will have errors well below 1\% for cardinalities beyond $10^{19}$\cite{Flajolet07hyperloglog:the}.

Unfortunately, it has a tendency to overestimate very low cardinalities, sometimes very significantly. To deal with this, the initial algorithm specifies a threshold below which $m \log{m/V}$(where $m$ is the number of entries, and $V$ is the number of empty counters) is a more accurate approximation. Heule et al\cite{HyperLogLogGoogle} took this a step further, and engineered a memory-compact implementation which uses empirical interpolation to reduce the error in small cardinalities. 

\section{Frequent Elements}
\label{sec:SpaceSaver}
With this, we seek to estimate the $k$ elements which occur most frequently in a data set\footnote{This problem is also referred to as the \emph{Heavy Hitters} or \emph{Icebergs}}. There are several algorithms which provide reasonable error rates.

One simple implementation is just random sampling; however, this provides relatively poor accuracy rates, and also has considerable space requirements. A superior approach is to the use the \emph{Space Saving} algorithm\cite{SpaceSaver}, which estimates the $k$ most frequent elements and their frequencies with a fixed storage cost.

The essence of the algorithm is to maintain a sorted list of triplets, of the form $(item,count,\epsilon)$. $count$ is an estimate of the frequency of $item$, and $\epsilon$ is a measure of how much $count$ could have been \emph{overestimated}. When a new item is visited, it is first compared against the list of currently stored elements. If it matches one of those, that counter is incremented. Otherwise, the item with the \emph{lowest} count is evicted, and the new elements is placed in its location. When the new element is placed, its $count$ is set to the previous elements $count+1$ and $\epsilon$ is set to the previous entry's $count$.

When the stream is finished, the data structure will hold the top $k$ most frequent elements, along with a frequency estimate which is guaranteed to be overestimated by no more than $\epsilon$.

\begin{algorithm}
				\KwData{let $M$ be a data set, and $m$ be a maximum number of triplets to store}
				\Begin{
								\For{$e \in M$}{
												\eIf{$e$ is monitored}{
																incrementCounter($e$) 
												}{
																set $e_{min} =$ the element with the least hits $min$ \;
																replace $e_{min}$ with $e$ \;
																incrementCounter($e$) \;
																set $\epsilon = min$
													}
								}
				}
				\caption{\emph{SpaceSaving} algorithm}
\end{algorithm}

The sort-order invariant is not required, but it makes the algorithm more efficient by reducing the number of comparisons that must be made. In fact, the \emph{SpaceSaving} algorithm as designed includes a special-purpose sorted data structure that allows for amortized constant-time replacement, which makes it considerably more efficient.

The error in \emph{SpaceSaving} comes from the possibility of missing elements. At any point in time, only the entry with the least number of hits is in danger of being evicted--thus, if $m$ separate entries are held, then at the end of the updating process, the last entry may be incorrect. Thus, it behooves one to collect a larger number of frequent elements than one needs to reduce this error.

Additionally, in very uniform data sets, there is a higher possibility for error. In general, one may detect this possibility by noting that the \emph{guaranteed count} $count-\epsilon$ is very small relative to the number of rows in the data set. In those cases, there is no such things as a "most frequent element"\footnote{Okay, unless the dataset is \emph{perfectly} uniform, there's always a most-frequent element. It's just that we don't care about them unless they are a \emph{lot} more frequent than everything else}, so using these estimates is not worthwhile anyway.

\section{Histograms}
\label{sec:Histograms}
High-quality histogram algorithms are an extremely complex area, and still a subject of active research in the academic literature. 

The main motivation behind a histogram is the need to approximate functions, in particular the distribution of data.

\begin{defn}
				Consider a data set $M$ which has a domain $D$. The \emph{distribution function}(sometimes also called the \emph{distribution}) $p$ is the function such that, for any $x \in M$, $p(x)$ is the number of times $x$ is found in $M$.
\end{defn}

On its own, the distribution is not the most effective tool. However, we can extend the definition slightly to be a more useful function:

\begin{defn}
				Consider a data set $M$ which has a domain $D$. For any $x \in M$, define $P(x) = \lbrace y \in M | y \leq x \rbrace$. The \emph{cumulative distribution function}(or \emph{cumulative distribution}) of $M$ is the function $c$ such that, for any $x \in M$, $c(x) = \left\| P(x) \right\|$ is the number of elements $y \in M$ such that $y \leq x$. 
\end{defn}

Note, for the more mathematically inclined, that it is always possible to construct $c(x)$ given $p(x)$--simply define $c(x) = \int_{min}^x p(x) dx$, where $min$ is the minimum possible value in $D$.

The cumulative function is particularly powerful in query optimizations, as it allows us to estimate the output size of qualified queries. However, constructing $p(x)$ or $c(x)$ exactly are $O(n)$ algorithms--we must essentially scan all elements and count them\footnote{Technically, if the column is a primary key, then we don't need to scan all data elements,but that is a very special case, which isn't very interesting}. Further, keeping a distribution for all elements precisely may require excessively large volume of resources (since all unique elements must be kept). 

However, if we were willing to sacrifice perfect accuracy in exchange for space, we can construct an approximation to either the distribution or the cumulative distribution which allows us to estimate output sizes without requiring data access.  The question now becomes: what type of approximate function should we construct?

The simplest solution is to approximate the distribution function $p(x)$ with a piecewise-constant function $H(x)$. To construct such a function, we first select $B+1$ distinct values $s_i$($i = 0,...,B$) from the domain $D$, called the \emph{boundary points}. Then we construct $B$ \emph{buckets} $b_i$, where each bucket is responsible for a range of values, and has a constant value--e.g. $b_i = ([s_i,s_{i+1}),c_i)$, where $c_i$ is a counter. Then, as each element in the data set is visited, the appropriate bucket's counter is incremented. $H(x)$ is the piecewise-constant function that is $c_i$ on the interval $[s_i,s_{i+1})$, which we call the \emph{histogram} of $M$.

It is a straightforward manner to approximate the number of elements which match $x$ using $H(x)$--merely find the bucket $b_i$ such that $s_i \leq x < s_{i+1}$, and $c_i$ is the estimate. Of course, there is some error in this estimate--in the worst case, $x$ is not present in the data set at all, in which case we are off by $c_i$; there are ways of reducing this error\footnote{such as using the cardinality as well as the count}, but the error cannot be totally removed.

$H(x)$ can also be used to approximate the cumulative distribution. First, define the function $S(i) = \sum_{j=0}^{i-1} c_i$ to be the sum of all buckets whose boundary is strictly less than $s_i$. Given $x$, find the bucket $b_i$ such that $s_i \leq x < s_{i+1}$. Then, compute $S(i)$. This sum approximates $c(s_i)$, so we need to also include the contribution of ranges between $s_i$ and $x$. To do this, we note that $S(i+1) = S(i) + c_i$; thus, we have two points on our graph: $(s_i,S(i))$ and $(s_{i+1},S(i) +c_i)$. Two points is enough for a straight line, so we compute the linear function $f(y) = ay + b$ which passes through those two points, and then compute $f(x)$. $f(x)$ is then a reasonable approximation of the contribution of elements between $s_i$ and $x$, so we know that $c(x) = S(i) + f(x)$.

								This technique is called \emph{linear interpolation}, and it also introduces some error into the estimate. The worst case occurs when no elements between $s_i$ and $s_{i+1}$ are present except for $s_i$. In this case, $f(x)$ will always overestimate by $f(x) - c_i$, which will grow linearly as $x$ goes from $s_i$ to $s_{i+1}$. This means that the error is bounded by the error in interpolating a single bucket, which can be controlled by appropriate choices for the boundary points.

								This raises a difficult question: how should we choose the boundary points $s_i$ so that our interpolation error is minimized? In fact, there are three major methods for doing so\footnote{although there are several additional minor approaches}:

\begin{enumerate}
				\item Equi-width
				\item Equi-depth
				\item V-Optimal
\end{enumerate}

\subsection{Equi-Width Histograms}
\label{sec:EquiWidth}
Equi-width histograms use the most obvious approach to choosing boundary points: simply divide the domain $D$ into $B$ equal intervals--choose $s_i$ such that $s_{i+1}-s_i$ is the same for all $i$.

\begin{exmp}[Choosing Equi-width boundary points]
				\label{exmp:EquiWidthBoundary}
				Suppose $D = [0,100)$ is the domain of possible values, and we wish to compute an Equi-width Histogram with $B = 10$ buckets. Then, we need 11 boundary points $s_i$ which cover the entirety of $D$, but maintain the constraint that $s_{i+1}-s_i$ are equal. Choose $s_i = 10i$. Then we have the boundary points $\lbrace 0,10,20,30,40,50,60,70,80,90,100 \rbrace$. This creates intervals $\lbrace ([0,10),[10,20),[20,30),[30,40),[40,50),[50,60),[60,70),[70,80),[80,90),[90,100)$ which satisfies the Equi-width constraint.
\end{exmp}

Equi-width histograms are extraordinarily simple to construct, as they require knowledge only of the domain of possible values, and the number of buckets that one wishes to construct. They are also linear, as long as the same boundary points are chosen by all subhistogram constructions (merging two Equi-depth histograms is simply adding the counters in each bucket).

However, Equi-width histograms have very poor error characteristics\cite{PiatetskyTuple}. In particular, they suffer badly when data is not uniformly distributed.

\begin{exmp}[Equi-width histogram with poor resolution of low cardinalities]
				\label{exmp:EquiWidthResolution}
				Suppose $D = [0,100)$ is the domain of possible values for data set $M$, and we compute $H(x)$ as in Example-\ref{exmp:EquiWidthBoundary}.
								Now suppose that $M_1 = {1,1,1,1,1,1,1,1,2,2,2,2,4,4,8}$. Then $b_0 = ([0,10),15)$, and $c_i = 0$ for $i \neq 0$. However, $H(4) = 15$, and the approximation of $c(4) = 6$. 
												Now, suppose that $M_2 = {1,1,1,1,2,2,2,2,4,4,8,8,8,8,8}$. Then $H(4) = 15$ and $c(4) = 6$ is the same as in the case of $M_1$, but the distribution is qualitatively different in both cases.
\end{exmp}

The problem in the above example is that the domain of possible data values is very large, but the domain of \emph{actual} values is very low. Unfortunately, there is no way to adjust Equi-width histograms based on the realities of data. Sure, we could correct the above example by using a different number of buckets (say, 100), but that would require a priori knowledge of the distribution which we do not possess.

We also cannot dynamically adjust our boundaries to allow for better resolution, because we must leave boundaries fixed for all data sets if we hope to make use of the linearity of Equi-width histograms.

As if that were not enough,As if that were not enough,  Equi-width histograms allow a high \emph{variance} between items within the same bucket--that is, it's possible that a single element in the bucket could have many, many more elements than the others within the bucket, which means that the linear interpolation would have higher error than it could otherwise have.

\subsection{Equi-Depth Histograms}
\label{sec:EquiDepth}
When considering Example-\ref{exmp:EquiWidthResolution}, we notice that the vast majority of the buckets are empty, while one bucket has all the data. This is both erroneous, and wasteful--we are using resources for a bunch of empty buckets, while simultaneously losing needed resolution. This leads us to believe that if we could somehow adjust this strategy so that we never have empty buckets, we could have a better histogram. Equi-depth histograms attempt to do just that.

In point of fact, Equi-depth histograms impose no restrictions of the domain of data, but rather on the \emph{depth} of each bucket. Given a data set $M$, we choose $B+1$ boundary points $s_i$ such that $c_i$ is (approximately) the same for all buckets.

\begin{exmp}[Constructing an EquiDepth Histogram]
				\label{exmp:EquiDepthHistogram}
Suppose that $M =\lbrace 1,1,1,1,1,1,1,1,2,2,2,2,4,4,8\rbrace$ is the data set of interest, and we wish to construct up to 5 buckets. We must choose boundaries such that $c_i$ remains as close to one another as possible. We start by defining a single bucket $([1,9),15)$. Since we want up to 5 buckets, we divide the bucket space in 2 to form 2 buckets: $b_0 = ([1,4),12)$ and $b_1 = ([4,9),3)$. Since $c_0 > c_1$, we subdivide $b_0$ again, to form $b_0 = ([1,2),12),b_1 = ([2,4),4),b_2=([4,9),3)$. Since we cannot subdivide $b_0$ any further, we are done. As close as possible (given this distribution), we have equal depths on our buckets (note that $c_1 \approx c_2$).
\end{exmp}

The algorithm used in Example-\ref{exmp:EquiDepthHistogram} is not the most efficient, as it requires multiple passes through the data to recompute frequencies for each bucket. However, there are a number of algorithms which have been devised that can compute \emph{approximate} Equi-depth histograms in a single pass; most notably, the algorithms described in \cite{MousaviEquiDepthDataStreams} and \cite{GibbonsFastIncremental}.

Equi-depth histograms eliminate the resolution error that is present in Equi-width histograms, by ensuring that the heights of each individual buckets are bounded, and also because \emph{possible} values are ignored--only values which are actually present are considered\footnote{In practice, we can make an Equi-depth histogram cover the entire domain by placing zeros outside the observed entries}.

Most database products use Equi-depth histograms(see Appendix-\ref{sec:OtherDBs}), because their construction is relatively easy, and the accuracy is fairly acceptable for query optimizer techniques. However, Equi-depth histograms are \emph{not} linear--we cannot merge two Equi-depth histograms together without violating the Equi-depth constraint

\begin{exmp}[Merging two Equi-depth Histograms introduces error]
				Consider two Histograms $H_1(x) = \lbrace ([0,10),10),([10,30),9),([30,60),11) \rbrace$ and $H_2(x) = \lbrace ([0,20),20),([20,25),19) \rbrace$. In order to merge $H_1(x)$ and $H_2(x)$, we must convert $H_2$ to fit the same bounds as $H_1$. To do this, we use linear interpolation to obtain $H_2(x) = \lbrace ([0,10),10),([10,20),10),([20,25),19) \rbrace$ and $H_1(x) = \lbrace ([0,10),10),([10,20),9/2),([20,25),9/4),([25,30),9/4),([30,60),11) \rbrace$. Combining these generates $H_m(x) = \lbrace ([0,10),20),([10,20),29/2),([20,25),85/4),([30,60),11) \rbrace$. This merged histogram does not satisfy the Equi-depth constraint.
\end{exmp}

As a result, we can only construct an Equi-depth histogram if we don't wish to keep region-level histograms; since that is an important feature, SpliceMachine will need to look further than simple Equi-depth histograms.

\subsection{V-Optimal Histograms}
\label{sec:VOptimal}
Even if Equi-depth histograms were linear, they are subject to unusual errors within a single bucket. In particular, if there is one value within the bucket which has the majority of the recorded instances in the data set, it introduces a more significant error when estimating other values. This is because the \emph{variance} between any two elements in the same bucket is not controlled.

\emph{V-Optimal} histograms attempt to correct this error by choosing boundaries that minimize the variance between items within the same bucket. This means that linear interpolation is essentially guaranteed to have small error, since only values which are present will be accounted for. It can be shown that, even in the worst case, V-Optimal histograms are still very accurate for selectivity estimation\cite{JagadishOptimalHistograms}.

Unfortunately, computing a V-Optimal histogram precisely is a complex problem, requiring $O(n^2B)$ operations\cite{JagadishOptimalHistograms}. On the other hand, computing an \emph{approximate} V-Optimal histogram can be performed in $O(n)$ steps\cite{GuhaApproximation,GuhaREHIST}. 

None of these algorithms, however, can make V-Optimal histogram functions linear. Therefore, even though they are superior in error and estimation properties to Equi-depth histograms, they are equally useless for SpliceMachine.

\subsection{Wavelet Histograms}
\label{sec:Wavelets}
We are stuck between a rock and a hard place. On the one hand, We know that Equi-depth and V-Optimal histograms provide the best accuracy for our purposes. On the other hand, we are confined to using an Equi-width histogram because it is the only linear option. We must choose to either sacrifice the quality of our estimations so that we can easily merge together different regions, or we must sacrifice our desire to maintain independent regions so that we can maintain quality histograms. Neither option is very appealing.

Thankfully, if we reconsider our approach to the problem, we can devise an alternative that can be very accurate, but is also linear, and therefore useful.

First, we know that what we \emph{truly} want is not a histogram. Instead, we only desire a function $w(x)$ that is an accurate approximation of $p(x)$ \emph{or} $c(x)$, since we should be able to easily convert between $c(x)$ and $p(x)$. Histogramming is a process of making $w$ a piecewise-constant function, but that isn't the only type of function available to us.

\emph{Wavelets} provide us with a technique for generating an approximation function $w$ which is \emph{not} piecewise-constant, but is nonetheless an accurate approximation of $c(x)$.

Wavelets as a mathematical subject are beyond the reach of this document--for a detailed analysis, see \cite{HernandezWavelets}. At the very roughest level, however, wavelet transforms treats data analogously to a radio \emph{signal}. In any signal, there are frequencies which contain information, and frequencies which are present as noise. A Wavelet transform seeks to remove the frequencies which do not significantly contribute to the signal, leaving behind only those frequencies which are very important\footnote{In a way, wavelets act as a form of lossy compression of the data stream}.

By definition and choice, Wavelet functions are linear, so any approximation that we make using wavelets can satisfy our desire for merging. The real question then is twofold: 

\begin{enumerate}
				\item How is a wavelet approximation function $\psi(x)$ constructed?
				\item How accurate are wavelet approximations
\end{enumerate}

In \cite{MatiasWaveletHistogram},Matias et al show that the accuracy of any wavelet transform is equivalent in quality to that of a V-Optimal histogram(and are in fact experimentally shown to be superior). However, the algorithm used requires $O(n)$ space, and $O(n)$ time to update, which is excessive for our purposes.

Gilbert et al, in \cite{GilbertSurfing} and \cite{GilbertWaveletHistograms}, describe a data structure known as an \emph{Array Sketch} that uses $O(\log^2{n})$ space to store, and has constant-time update performance\footnote{Readers who are able to interpret those articles should be rewarded as excellent mathematicians in their own right}. However, constructing a wavelet histogram from an Array Sketch is a very expensive operation\cite{CormodeGroupedCountSketch}.

\subsubsection{GroupedCount Sketch}
\label{sec:GroupedCountSketch}
In more recent efforts, Cormode et al have managed to construct an algorithm for computing a wavelet histogram that is inexpensive to maintain and \emph{also} inexpensive to reconstruct\cite{CormodeGroupedCountSketch}.

This algorithm essentially maintains $\log{n}$ separate \emph{GroupedCountSketch}(\emph{GCS}) data structures. The \emph{GCS} is a three-dimensional array with parameters $t,b$ and $c$, which determine the level of accuracy. As elements are visited, a selection of \emph{GCS} structures are updated, each in an $O(1)$ update operation (using uniform hash functions); the entire process requires $O(\log{n})$ to update. Then, when the wavelet is desired, a threshold is chosen, and only elements from the \emph{GCS} which exceed this threshold are chosen, thus selecting the most significant wavelet coefficients to use. 

The GroupedCountSketch uses relatively little memory to compute, and is also linear and highly accurate. Thus, it is the Histogramming choice for SpliceMachine. In particular, it directly estimates the cumulative distribution $c(x)$.


				\section*{Appendix: Statistics engine in Lassen}
				\label{sec:Lassen}
				This appendix contains a brief mapping between the features in this document and those described the in the PRD, along with information about its presence (or absence) in the Lassen release of SpliceMachine.

\section{PRD features}
This is an overview of the features which were explicitly requested as part of the PRD document. Most are already completed; those which are not were likely deferred because of a change in design course.
\subsection{TableStatistics} 
	Completed as designed in this document. Contents of \texttt{systablestatistics} and \texttt{systablestats} are complete, with the exception of Query counts
\subsection{ColumnStatistics}: Completed as designed in this document. Contents of \texttt{syscolumnstats} and \texttt{syscolumnstatistics} are complete.
\subsection{IndexStatistics}: Are merged with TableStatistics and ColumnStatistics. Index fetch latency (the latency to fetch a single base table for a given index row) is stored as \texttt{remote\_read\_latency} in \texttt{systablestats}
\subsection{Update and Report Statistics}
Updates for synchronous collections are completed for individual tables and schemas. All other calls described in \ref{sec:ManualCollection} were deferred for lack of time, although they would be trivial to implement (requiring only an additional Stored Procedure).
\subsection{Stats on a Region} Manual collection is supported at the engine layer, but no Stored Procedure yet exists. It would be trivial to complete this(require $<$ 1 day).
\subsection{Stats on a greater sample set of data} Does not match with this design, as it refers to sampling data points (an infeasible proposition over an LSM tree structure such as HBase)
\subsection{Failure Semantics} Are supported within the engine as proper error messages, similarly to any other Stored Procedure. Collections are transactional, so they are rolledback on failure similarly to insertion or imports.
\subsection{Resource Management} The Task framework is used to manage statistics collections. Rate limiting was deemed unnecessary for forcible manual collections, but will be present for automatic collections when implemented

\section{Implicit PRD features}
These are features mentioned in the PRD, but which are not given a relative priority.
\subsection{Historical Statistics} Deferred due to complexity, and a relative lack of interest in the feature overall.

\section{Design document Features}
These are features which are present in the design document, but are not explicit features in the PRD. The difficulty of each task varies.

\subsection{AutoCollect} Due to confusion about its purpose and design, this was deferred to post-Lassen.
\subsection{Collection for Specific Columns} Is trivial to implement, but has not seen much demand, so was deferred
\subsection{Collection for Specific Key Range} Identical to "Stats on a region" in PRD design. Was deferred due to lack of interest in the feature.
\subsection{Stale-partition detection} Was deferred to avoid scope-creep. Necessary for AutoCollect. Would require ~ 1-2 days to fully implement.
\subsection{Histograms} Very difficult algorithmically. Will require significant resources to implement and test for correctness. Deferred as a result. 

\section{Other features}
These are features which are not present in the design document or in the PRD document, but which have arisen in discussions. Addressing these will require a design (with an associated update to this document) plus implementation effort, and should be considered at the upper edge of difficulty unless otherwise mentioned.

\subsection{Finer-grained partitioning} Separate a region into smaller partitions, to allow more parallelism and to prevent traversing the entire region's data set on collection. This was deferred due to relative complexity.
\subsection{Enable/Disable statistics for all columns} Trivial to implement (<1 day). Various parties expressed interest as well. Good candidate for Lassen.

\end{subappendices}

\clearpage
\printbibliography

\end{document}
