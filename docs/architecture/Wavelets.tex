\documentclass[10pt]{amsart}
\usepackage[top=1.5in, bottom=1.25in, left=1.25in, right=1.25in]{geometry}

\usepackage{mathtools}
\usepackage{hyperref}
\usepackage{amsmath}
\usepackage{amsthm}
\usepackage{algorithm}
\usepackage{algpseudocode}
\usepackage{listings}

\newtheorem{theorem}{Theorem}[section]
\newtheorem{defn}{Definition}[section]
\newtheorem{exmp}{Example}[section]

\begin{document}
\title{Transactions in SpliceMachine}
\author{Scott Fines}

Consider a \emph{signal} $a$ to be the stream of incoming integers falling in the range of $[0,N)$ for some $N$\footnote{We will extend this argument to negative numbers after first deriving the proper formula}, arriving in any order.

\begin{exmp}[Example signal]
\begin{displaymath}
a = [2,2,0,2,3,5,4,4] (N=8)
\end{displaymath}
\end{exmp}
A signal is just a stream of numbers; it is more helpful for us to consider the stream as a matrix, which is as follows:

\begin{displaymath}
\begin{array}{ccccccccc}
	0 & 0 & 1 & 0 & 0 & 0 & 0 & 0 & 0 \\
	0 & 0 & 1 & 0 & 0 & 0 & 0 & 0 & 0 \\
	1 & 0 & 0 & 0 & 0 & 0 & 0 & 0 & 0 \\
	0 & 0 & 1 & 0 & 0 & 0 & 0 & 0 & 0 \\
	0 & 0 & 0 & 1 & 0 & 0 & 0 & 0 & 0 \\
	0 & 0 & 0 & 0 & 0 & 1 & 0 & 0 & 0 \\
	0 & 0 & 0 & 0 & 1 & 0 & 0 & 0 & 0 \\
	0 & 0 & 0 & 0 & 1 & 0 & 0 & 0 & 0 
\end{array}
\end{displaymath}
Where each row is a counter for the value that the next element in the stream is at. For example, the first row increments the counter in position 3 by 1, which is an indication that we see the value $2$.

When we look at the stream like this, we can see that we are contributing in only specific locations. We proceed with this line of reasoning.

We define a \emph{dyadic range} as
\begin{defn}[Dyadic Range]
a \emph{dyadic range} is as follows:

\begin{displaymath}
d_{l,k} = [k2^{\lg{N}-l},(k+1)2^{\lg{N}-l})
\end{displaymath}
for $l=0,...,\lg{n}$ and $k=0,...,2^l-1$
\end{defn}

We then consider the vector

\begin{displaymath}
\phi_{l,k}[i] = \begin{cases} 2^{l-lgN} &\mbox{if } i \in d_{l,k} \\ 0 & \mbox{otherwise} \end{cases}
\end{displaymath}
for $i=0,...N-1$. This vector essentially "picks out" the contribution of a given element in the signal. We 
\end{document}
